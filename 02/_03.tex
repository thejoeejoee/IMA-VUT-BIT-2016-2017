\section{3. úkol}
\subsection*{Zadání}
Pomocí derivace nebo integrace najděte součet řady a vyšetřete její obor konvergence.

\begin{eqnarray}
\sum\limits_{n=1}^\infty \dfrac{(x-3)^{2n}}{2n} \label{eqn:infty}
\end{eqnarray}


\subsection*{Rozbor příkladu}
Máme vyšetři obor konvergence a majít součet nekonečné řady \ref{eqn:infty}. Najít obor konvergence znamená najít všechna $x$, pro která po dosazení vznikne konvergující číselná řada. Toho dosáhneme za pomoc podílového kritéria. D8le máme vypočítat součet řady za pomoci derivace nebo integrace. Řadu si upravíme tak, aby $n$ začínalo na hodnotě nula a provedeme integraci řady.

\subsection*{Řešení}
\subsubsection{Výpočet oboru konvergence}

\begin{displaymath}
\lim_{n \rightarrow \infty} \dfrac{\big |\,c_{n+1}(x - x_o)^{n + 1}\,\big |}{\big | \,c_n(x - x_o)^{n} \, \big |} = 
\lim_{n \rightarrow \infty}  \dfrac{\dfrac{\big |\,x - 3\, \big|^{2(n+1)}}{2(n+1)}}{\dfrac{\big | \, x-3\, \big |^{2n}}{2n}} =
\lim_{n \rightarrow \infty} \dfrac{2n \big |\,x-3\,\big |^{2n+2}}{2(n+1)\big |\,x-3\,\big |^{2n}} = 
\lim_{n \rightarrow \infty}  \frac{n\big |\,x-3\,\big |^{2}}{n+1} = 
\end{displaymath}

\begin{displaymath}
= \big |\,x-3\,\big |^{2}\cdot\lim_{n \rightarrow \infty}  \frac{n}{n+1} =
\big |\,x-3\,\big |^{2}\cdot\overbrace{\lim_{n \rightarrow \infty}  \frac{n}{n+1}}^{\rightarrow 1} = \big |\,x-3\,\big |^{2}
\end{displaymath}
\vspace{18px}

\noindent Budeme hledat takové hodnoty, pro které je výraz $\big |\,x-3\,\big |^{2} < 1$

\begin{displaymath}
\big |\,x-3\,\big |^{2} < 1\Leftrightarrow \big |\,x-3\,\big | < 1 \Leftrightarrow x \in (2, 4)
\end{displaymath}

\noindent Dále musíme prověřit krajní body intervalu.
\vspace{10px}

\noindent Pro $c = 4$ dostaneme číselnou řadu

\begin{eqnarray}
\sum_{n=1}^{\infty} \dfrac{1^{2n}}{2n} = \sum_{n=1}^{\infty} \dfrac{1}{2n}
\label{eqn:odvozenaSuma}
\end{eqnarray}

\noindent Jedná se o řadu, o které víme, že diverguje.
\vspace{10px}

\noindent Pro $c = 2$ dostaneme číselnou řadu

\begin{displaymath}
\sum_{n=1}^{\infty} \dfrac{(-1)^{2n}}{2n} = \sum_{n=1}^{\infty} \dfrac{1}{2n}
\end{displaymath}
\noindent Což diverguje, viz. \ref{eqn:odvozenaSuma}

\vspace{10px}

\noindent Obor konvergence je tedy interval $(2, 4)$

\subsubsection{Výpočet součtu nekonečné řady}

\begin{displaymath}
\sum_{a = 1}^{\infty} \dfrac{(x - 3)^{2n}}{2n} = \sum_{a = 0}^{\infty} \dfrac{(x - 3)^{2(n + 1)}}{2(n + 1)} = \int_{3}^{x} \sum_0^{\infty} (x - 3)^{2n - 1} =
\end{displaymath}

Vnitřní řada je geometrická řada, pro kterou platí
\begin{displaymath}
a1 = x - 3 \qquad q = (x - 2)^2
\end{displaymath}

\begin{displaymath}
= \int_{3}^{x} \dfrac{x - 3}{1 - (x - 3)^2}dx =  \int_{3}^{x} \dfrac{x - 3}{- x^2+6x-8}dx = - \frac{1}{2}  \int_{3}^{x} \dfrac{-2x + 6}{- x^2+6x-8}dx =- \frac{1}{2} \Big [  \mbox{ln}(-x^2 + 6x - 8) \Big ]_3^x =
\end{displaymath}

\begin{displaymath}
= - \frac{1}{2} \mbox{ln}(-x^2 + 6x - 8)
\end{displaymath}



