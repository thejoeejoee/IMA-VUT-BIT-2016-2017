\section{3. úkol}
\subsection*{zadání}
Pomocí derivace nebo integrace najděte součet řady a vyšetřete její obor konvergence.

\begin{eqnarray}
\sum\limits_{n=1}^\infty \dfrac{(x-3)^{2n}}{2n} \label{eqn:infty}
\end{eqnarray}


\subsection*{Rozbor příkladu}
Máme vyšetři obor konvergence a majít součet nekonečné řady \ref{eqn:infty}. Najít obor konvergence znamená najít všechna $x$, pro která po dosazení vznikne konvergující číselná řada. Toho dosáhneme za pomoc podílového kritéria.

\subsection*{Řešení}
Výpočet oboru konvergence:

\begin{displaymath}
\lim_{n \rightarrow \infty} \dfrac{\big |\,c_{n+1}(x - x_o)^{n + 1}\,\big |}{\big | \,c_n(x - x_o)^{n} \, \big |} = 
\lim_{n \rightarrow \infty}  \dfrac{\dfrac{\big |\,x - 3\, \big|^{2(n+1)}}{2(n+1)}}{\dfrac{\big | \, x-3\, \big |^{2n}}{2n}} =
\lim_{n \rightarrow \infty} \dfrac{2n \big |\,x-3\,\big |^{2n+2}}{2(n+1)\big |\,x-3\,\big |^{2n}} = 
\lim_{n \rightarrow \infty}  \frac{n\big |\,x-3\,\big |^{2}}{n+1} = 
\end{displaymath}

\begin{displaymath}
= \big |\,x-3\,\big |^{2}\cdot\lim_{n \rightarrow \infty}  \frac{n}{n+1} =
\big |\,x-3\,\big |^{2}\cdot\overbrace{\lim_{n \rightarrow \infty}  \frac{n}{n+1}}^{\rightarrow 1} = \big |\,x-3\,\big |^{2}
\end{displaymath}
\vspace{18px}

\noindent Budeme hledat takové hodnoty, pro které je výraz $\big |\,x-3\,\big |^{2} < 0$

\begin{displaymath}
\big |\,x-3\,\big |^{2} < 0\Leftrightarrow \big |\,x-3\,\big | < 0 \Leftrightarrow x \in (2, 4)
\end{displaymath}

\noindent Obor konvergence je tedy interval $(2, 4)$

