\section{1. úkol}
Vyšetřete lokální extrémy funkce $f(x) = \int\limits_0^x t(t-1)(t-5)\mathrm{d}t$.

\subsection*{Rozbor řešení}
Budeme hledat lokální extrémy funkce, kterou získáme zintegrováním zadaného určitého integrálu v~intervalu $\langle 0, x \rangle$.

\subsection*{Řešení}
\begin{align*}
	f(x) = \int_0^x t(t-1)(t-5)\mathrm{d}t & = \int_0^x (t^3 - 6t^2 + 5t)\mathrm{d}t = \Big[\frac{t^4}{4} - 2t^3 + \frac{5t^2}{2}\Big]_0^x \\
	f(x) & = \frac{x^4}{4} - 2x^3 + \frac{5x^2}{2} \\
	f'(x) & = (x^3 - 6x^2 + 5x)\mathrm{d}x \\
	f'(x) & = 0  \rightarrow (x^3 - 6x^2 + 5x) = x(x - 5)(x - 1) \\
	x_1 & = 0 \qquad x_2 = 1\qquad x_3 = 5 \\
\end{align*}

Nyní máme tři $x$, ve kterých první derivace určitého integrálu je rovna nule, tato $x$ tedy označíme jako $x$ podezřelá z~extrémů. Dosazením do $f(x)$ získáme hodnoty konkrétních určitých integrálů v~podezřelých $x$.

\begin{align*}
	f(x_1) = 0 \qquad f(x_2) = \frac{3}{4} \qquad f(x_3) = -31.25
\end{align*}

Nyní můžeme $x_1$ vyloučit, protože se nejedná o~lokální extrém ale pouze o~hodnotu podezřelou z~něj. Řešením je tedy lokální maximum v~bodě $x_2 = 1$, kde je funkční hodnota $\frac{3}{4}$ a lokální minimum v~bodě $x_3 = 5$ s~funkční hodnotou $-31.25$.