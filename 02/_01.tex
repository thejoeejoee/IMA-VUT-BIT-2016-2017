\section{1. úkol}
Vyšetřete lokální extrémy funkce $f(x) = \int\limits_0^x t(t-1)(t-5)\mathrm{d}t$.

\subsection*{Rozbor řešení}
Budeme hledat lokální extrémy funkce, která je dána hodnotou určitého integrálu  $\langle 0, x \rangle$. Zjistíme si tedy funkci, která určuje hodnotu tohoto integrálu, poté nalezneme podezřelé body a přidáme k nim krajní bod, ve kterém je hodnota integrálu 0, dosadíme do 

\subsection*{Řešení}

\noindent Nalezení funkce, která určuje hodnotu určitého integrálu
\begin{displaymath}
	f(x) = \int_0^x t(t-1)(t-5)\mathrm{d}t = \int_0^x (t^3 - 6t^2 + 5t)\mathrm{d}t = \Big[\frac{t^4}{4} - 2t^3 + \frac{5t^2}{2}\Big]_0^x = \frac{x^4}{4} - 2x^3 + \frac{5x^2}{2}
\end{displaymath}

\noindent Nalezení podezřelých bodů
\begin{displaymath}
	f'(x) = x^3 - 6x^2 + 5x
\end{displaymath}
\begin{eqnarray}
	f'(x) & = 0 \nonumber\\
    x^3 - 6x^2 + 5x & = 0 \nonumber \\
    x(x - 5)(x - 1) & = 0 \nonumber
\end{eqnarray}
\begin{displaymath}
	x_1 = 1\qquad x_2 = 5
\end{displaymath}

Nyní máme dvě hodnoty funkce, ve kterých může nabývat extrému. K těmto dvoum hodnotá ještě přičteme bod $x_3 = 0$

Nyní máme tři $x$, ve kterých první derivace určitého integrálu je rovna nule. U hodnot $x_2 \text{ a } x_3$ musíme ještě rozhodnout, zda se jedná o skutečné extrémy nebo jen inflexní body. Tedy označíme $x_2 \text{ a } x_3$ jako podezřelá z~extrémů.

\begin{eqnarray}
f'(0.5) = 1.125 \quad f'(2) = -6 \Rightarrow \mbox{Lokální maximum} \nonumber \\
f'(2) = -6 \quad f'(6) = 30 \Rightarrow \mbox{Lokální minimum} \nonumber
\end{eqnarray}

\noindent Dosazením do \( f(x) \) získáme hodnoty konkrétních určitých integrálů.

\begin{align*}
	f(x_1) = 0 \qquad f(x_2) = \frac{3}{4} \qquad f(x_3) = -31.25
\end{align*}

\subsection*{Závěr}
\noindent Funkce má v bodě $x = 0$ lokální minimum a jeho hodnota je $0$

\noindent Funkce má v bodě $x = 1$ lokální maximum a jeho hodnota je $\frac{3}{4}$

\noindent Funkce má v bode $x = 5$ lokální minimum a jeho hodnota je $-31.25$