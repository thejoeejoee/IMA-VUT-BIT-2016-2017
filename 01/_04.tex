\section{4. úkol}

\subsection{Zadání}
Načrtněte graf funkce $f$,
pro kterou platí: $\text{D}_{f} = \mathbb{R} – \{1\}$, pro $x = 1$ má nespojitost 2.druhu a následně platí:

Do obrázku nakreslete i asymptoty a tečny resp. polotečny ke grafu funkce v bodech $x = 0, x = 1$ a $x = –1$.

\begin{align*}
f(0) = f(-1) = 0\quad f'(0) = –2 &\qquad \lim_{x\rightarrow 1^{+}} f(x) = 2\qquad\quad\lim_{x\rightarrow –\infty} f(x) = 2\\
\lim_{x\rightarrow –1^{-}} f'(x) = – \infty&\qquad \lim_{x\rightarrow –1^{+}} f'(x) = \infty\qquad\lim_{x\rightarrow 1^{+}} f'(x) = – 2\\
f''(x) > 0;&\qquad\forall x \in (0, 1) \cup (1, \infty)\\
f''(x) < 0;&\qquad\forall x \in (–\infty, –1) \cup (–1,0)\\
\text{přímka}\ y = 2 - x &\qquad\text{je asymptota pro}\ x\rightarrow \infty
\end{align*}

\subsection{Řešení}
Po určení definičního oboru jsme si do grafu zakreslili funkční hodnoty v přímo zadaných bodech ($0\ \text{a} -1$). Poté jsme dle limit k $1^+$ a $-\infty$ zakreslili limitní hodnoty pro tyto hodnoty. Dle funkčních hodnot první derivace jsme zakreslili tečny v daných bodech, dle druhé derivace potom vyznačili v daných intervalech konvexnost, resp. konkávnost.

\begin{figure}[H]
	\centering
	\includesvg[width=\textwidth]{assets/graf}
	\caption{Nákres funkce $f$}
\end{figure}