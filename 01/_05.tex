\section{5. úkol}
\subsection{Zadaní}
 Najděte největší a nejmenší hodnotu funkce $f(x)=\sqrt[3]{(6x^2-x^3)} $ na intervalu $ \langle-2,9\rangle$.

\subsection{Rozbor příkladu}
Funkce může nabývat svého maxima a minima v~bodech, kdy je derivace rovna nule nebo v~bodech, kde není derivace definována. Vypočítáme proto první derivaci funkce a najdeme tyto body. K~bodům přidáme krajní body zadaného intervalu.

V~těchto bodech zjistíme funkční hodnoty, z~čehož zjistíme v~kterých bodech má funkce maximum a v~kterých bodech má funkce minimum.
\subsection{Výpočet}
Určíme definičný obor funkce: $D(f)=R$

Funkci máme zadanou s~odmocninami. Odmocniny si přepíšeme do tvaru mocnin.

\begin{displaymath}
f(x)=\sqrt[3]{(6x^2-x^3)}=(6x^2-x^3)^\frac{1}{3}
\end{displaymath}

Najdeme první derivaci funkce a upravíme ji na vhodný tvar tak, abychom snáze našli stacionární body a určili definiční obor $f'(x)$.

\begin{displaymath}
f'(x)=\left(\frac{1}{3}\cdot\frac{1}{\sqrt[3]{(6x^2-x^3)^2}}\right)\cdot(12x-3x^2)=\frac{3\cdot(4x-x^2)}{3\cdot\sqrt[3]{(6x^2-x^3)^2}}
\end{displaymath}
Jmenovatel zlomku nesmí být roven 0. Z~následující rovnice zjistíme body, kde není derivace definována. V~těchto bodech může také funkce nabývat svého maxima nebo minima (limita derivace může být $\pm \infty$.)
\begin{displaymath}
6x^2-x^3\neq0
\end{displaymath}
\begin{displaymath}
x^2(6-x)\neq0
\end{displaymath}
\begin{displaymath}
x\neq0 \wedge  x\neq6
\end{displaymath}
\begin{displaymath}
D(f')=R-\{0,6\}
\end{displaymath}
Derivaci položíme rovnu 0. To může nastat pouze v~případě, kdy čitatel zlomku je roven 0.
\begin{displaymath}
x\cdot(4-x) = 0 \Rightarrow x = 0 \vee x = 4
\end{displaymath}
K~stacionárním bodům přidáme ješte hraniční hodnoty. 
Podezřelé hodnoty tedy jsou body $ \in\{-2,0,4,6,9\} $
\begin{displaymath}
f(0) = 0,\quad f(4) \doteq 3.17,\quad f(6) = 0,\quad f(-2) \doteq 3.17,\quad f(9) \doteq 3.12
\end{displaymath}

Maximum na intevalu $ \langle-2,9\rangle$ je v~bodech $x=4$ a $x=-2$ jeho hodnota je přibližně $3.17$.

Minimum na intevalu $ \langle-2,9\rangle$ je v~bodě $x=0$ a $x=6$ a jeho hodnota je $0$.
