\section{5. úkol}

\subsection{Zadanie}
 Najděte největší a nejmenší hodnotu funkce $f(x)=\sqrt[3]{(x+1)^2} -\sqrt[3]{(x-1)^2}$ na intervale $ <-2,2>$.
\subsection{Výpočet}
\begin{align*}
f'(x)=\left(\frac{2}{3}*\frac{1}{\sqrt[3]{(x+1)}}\right)*1\
-\left(\frac{2}{3}*\frac{1}{\sqrt[3]{(x-1)}}\right)*1
\end{align*}

Upravíme na vhodný tvar a nájdeme stacionárne body
\begin{align*}
f'(x)=\frac{2*(\sqrt[3]{(x-1)}-\sqrt[3]{(x+1))}}{3*\sqrt[3]{(x+1)}*\sqrt[3]{(x-1)}}
\end{align*}

\begin{center}
Body $x=1 \wedge  x=1$ su stacionarne body.
\end{center}

Zistíme kedy sa derivácia rovná 0. 

\begin{align*}
2*(\sqrt[3]{(x-1)}-\sqrt[3]{(x+1))}=0
\end{align*}

Upravíme podľa vzorca $ a^2-b^2$\\
\begin{align*}
\left(\sqrt[6]{(x-1)}-\sqrt[6]{(x+1)}\right)*\left(\sqrt[6]{(x-1)}+\sqrt[6]{(x+1)}\right)=0 
\end{align*}

$$x-1=x+1 \Rightarrow -1\neq=1$$
$$x-1=-x-1 \Rightarrow 2x=0$$
K stacionárnym bodom pridáme ešte hraničné hodnoty$ <-2,2>$.\\
Dostaneme body $-2,-1,0,1,2$. Tieto body dosadíme do základnej funkcie.
\begin{center}
$f(-2)=1-\sqrt[3]{9} \hspace{10mm} f(-1)=-\sqrt[3]{4} \hspace{10mm}  f(0)=0 $
\end{center}
\hspace{19mm} $f(1)=\sqrt[3]{4} \hspace{20mm} f(2)=\sqrt[3]{9}-1$

\begin{center}
\large{Maximum na intevale $ <-2,2>$ je v bode $x=1$ \\
 Minimum je v bode $x=-1$}
\end{center}
