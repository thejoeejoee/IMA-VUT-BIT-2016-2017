\section{3. úkol}
\subsection{Zadání}

Na grafu funkce \(f(x) = x^2 - x\) najděte bod, který má nejkratší vzdálenost od bodu \(A = [0, 1]\). Řešte jako úlohu na extrém.

\subsection{Řešení}

Určíme si účelovú funkciu, ktorej funkčné hodnoty budú nadobúdať hodnoty vzdialenosti od daného bodu A.

\begin{displaymath}
u(x)=\sqrt[2]{(x-x_0)^2 + (y-y_0)^2}
\end{displaymath}
\begin{displaymath}
u(x)=\sqrt[2]{(x-0)^2 + (x^2-x-1)^2}
\end{displaymath}
\begin{displaymath}
u(x)=(x^4 -2x^3 + 2x + 1)^{\frac{1}{2}}
\end{displaymath}

Zo zadania vyplýva, že máme nájsť x, ktorého funkčná hodnota je najmenšia, čiže najkratšia vzdialenosť. \\
K nájdeniu globálneho minima funkciu zderivujeme.

\begin{displaymath}
u'(x)=\frac{1}{2}(x^4 -2x^3 + 2x + 1)^{-\frac{1}{2}}\cdot(4x^3 - 6x^2 + 2)
\end{displaymath}

\begin{displaymath}
u'(x)=\frac{2x^3 - 3x^2 + 1}{\sqrt[2]{x^4 -2x^3 + 2x + 1}}
\end{displaymath}

Výraz v menovateli nebude nikdy nadobúdať hodnoty 
\begin{displaymath}
k(x)=x^4 -2x^3 + 2x + 1
\end{displaymath}
Nebudeme počítať deriváciu s odmocninou, tá umiestnenie minima nezmení
\begin{displaymath}
k'(x)=4x^3 -6x^2 + 2
\end{displaymath}
Nájdeme dva podozrivé body: 1 a $-\frac{1}{2}$
Keď overíme monotónnosť funkcie, v x=1 je len inflexný bod a v x=$-\frac{1}{2}$ globálne minimum.

Ak dosdíme naše minimum do funkcie k, zistíme že pod odmocninou bude minimálne hodnota \(\frac{5}{16}\), čiže náš menovateľ je vždy definovaný na celom definičnom obore.
\begin{displaymath}
k(-\frac{1}{2})=-\frac{1}{2}^4 -2-\frac{1}{2}^3 + 2-\frac{1}{2} + 1
\end{displaymath}

\begin{displaymath}
k(-\frac{1}{2})=\frac{5}{16}
\end{displaymath}

Čitateľa si upravíme pomocou Hornerovej schémy na prívetivejší tvar

\begin{table}[!h]
\centering
\begin{tabular}{l||l|l|l|l|l}
	 & 4 & -6 &  0 & 2 &    \\ \hline\hline
   1 & 4 & -2 & -2 & 0 & OK \\ \hline
   1 & 4 &  2 &  0 &   & OK \\
\end{tabular}
\caption{Hornerovo schéma pro rozklad $4x^3 - 6x^2 + 2$}
\end{table}

Dostávame v čitateli

\begin{displaymath}
l(x)=(x-1)^2\cdot (4x +2)
\end{displaymath}

Nájdeme si podozrivé body, znovu použijeme Hornerovo schéma

\begin{table}[!h]
\centering
\begin{tabular}{l||l|l|l|l|l}
	 & 2 & -3 &  0 & 1 &    \\ \hline\hline
   1 & 2 & -1 & -1 & 0 & OK \\ \hline
   1 & 2 &  1 &  0 &   & OK \\
\end{tabular}
\caption{Hornerovo schéma pro rozklad $3x^3 - 3x^2 + 1$}
\end{table}

\begin{displaymath}
l(x)=(x-1)^2\cdot (2x +1)
\end{displaymath}

Podozrivé body: 1, \(-\frac{1}{2}\)

Keď vyšetríme monotónnosť, tak sa z 1 stane inflexný bod a z \(-\frac{1}{2}\) sa stane globálne minimum.
Už stačí vypočítať funkčnú hodnotu v globálnom minime.

\begin{displaymath}
f(-\frac{1}{2})= x^2 - x
\end{displaymath}

\begin{displaymath}
f(-\frac{1}{2})= \frac{3}{4}
\end{displaymath}

Najbližší bod k bodu A, z funkcie f je : [\(-\frac{1}{2}\),\(\frac{3}{4}\)]
 
\begin{tikzpicture}[scale=4]
	\draw[->] (-0.3,0) -- (1.3,0) node[right] {$x$};
	\draw[->] (0,-0.5) -- (0,0.5) node[above] {$y$};
	\draw[blue] plot[samples=2000,domain=-0.25:1.25] function {x**2 - x};
	\draw[red] (0,0.3) node[circle,fill,inner sep=1pt,label=right:$A$]{};
\end{tikzpicture}