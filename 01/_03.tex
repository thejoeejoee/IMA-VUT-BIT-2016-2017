\section{3. úkol}
\subsection{Zadání}

Na grafu funkce \(f(x) = x^2 - x\) najděte bod, který má nejkratší vzdálenost od bodu \(A = [0, 1]\). Řešte jako úlohu na extrém.

\subsection{Rozbor příkladu}

Prvně nadefinujeme účelovou funkci, u které funkční hodnota je vzdálenost mezi bodem \(A\) a bodem ležícím na funkci \(f\).
Následně budeme hledat minima naší účelové funkce \(U\). A globální minimum funkce \(U\) bude souřadnice na ose \(x\) hledaného bodu.

\subsection{Řešení}
\begin{figure}[H]
	\centering
	\begin{tikzpicture}[scale=4]
		\draw[->] (-0.3,0) -- (1.3,0) node[right] {$x$};
		\draw[->] (0,-0.5) -- (0,0.5) node[above] {$y$};
		\draw[blue] plot[samples=2000,domain=-0.25:1.25] function {x**2 - x};
		\draw[red] (0,0.3) node[circle,fill,inner sep=1pt,label=right:$A$]{};
	\end{tikzpicture}
	\caption{Nákres funkce $f$ a bodu $A$}
\end{figure}

Určíme si účelovou funkci \(U\), jejiž funkční hodnoty nabývají hodnoty vzdálenosti mezi body A a bodem ležící na funkci \(f\).

\begin{displaymath}
u(x)=\sqrt[2]{(x-x_0)^2 + (y-y_0)^2}
\end{displaymath}
\begin{displaymath}
u(x)=\sqrt[2]{(x-0)^2 + (x^2-x-1)^2}
\end{displaymath}
\begin{displaymath}
u(x)=(x^4 -2x^3 + 2x + 1)^{\frac{1}{2}}
\end{displaymath}

Ze zadání vyplývá, že máme najít takové \(x\), jehož funkční hodnota je nejmenší, respektive má nejkratší vzdálenost od bodu \(A\). \\
K nalezení minima zderivujeme funkci.

\begin{displaymath}
u'(x)=\frac{1}{2}(x^4 -2x^3 + 2x + 1)^{-\frac{1}{2}}\cdot(4x^3 - 6x^2 + 2)
\end{displaymath}

\begin{displaymath}
u'(x)=\frac{2x^3 - 3x^2 + 1}{\sqrt[2]{x^4 -2x^3 + 2x + 1}}
\end{displaymath}

Podezřelé body jsou v místech, kdy funkce nabývá nulové nebo nedefinované hodnoty. 
Jediná situace, kdy bude mít funkce nedefinované hodnoty, je v případě nulového jmenovatele nebo záporné hodnoty pod druhou odmocninou.
K zjištění, zda bude někdy výraz pod odmocninou někdy záporný či nulový, budeme hledat opět extrémy výrazu pod odmocninou (dále funkce \(k\)).

%Výraz v menovateli nebude nikdy nadobúdať hodnoty 
\begin{displaymath}
k(x)=x^4 -2x^3 + 2x + 1
\end{displaymath}
%Nebudeme počítať deriváciu s odmocninou, tá umiestnenie minima nezmení
Opět funkci zderivujeme a najdeme pouze nulové body, protože tato funkce má \(\implies D(f) = \mathbb{R}\).
\begin{displaymath}
k'(x)=4x^3 -6x^2 + 2
\end{displaymath}

Výraz si upravíme pomocí Hornerova schématu, abychom snadněji zjišťovali snadněji nulové body.
\begin{table}[!h]
\centering
\begin{tabular}{l||l|l|l|l|l}
	 & 4 & -6 &  0 & 2 &    \\ \hline\hline
   1 & 4 & -2 & -2 & 0 & OK \\ \hline
   1 & 4 &  2 &  0 &   & OK \\
\end{tabular}
\caption{Hornerovo schéma pro rozklad $4x^3 - 6x^2 + 2$}
\end{table}

Po úpravě dostaneme \(k'(x)=(x-1)^2(4x + 2)\), najdeme podezřelé body, které jsou: $1$ a $-\frac{1}{2}$\\.

\begin{figure}[H]
	\centering
	\includesvg[width=.45\textwidth]{assets/osa}
	\caption{Monotónost funkce \(k\)}
\end{figure}

Po ověření monotónosti zjistíme, že v $x=1$ je pouze inflexní bod a v $x=-\frac{1}{2}$ minimum, 
které můžeme označit za globální, protože na intervalu \(\langle-\frac{1}{2},+\infty\rangle\) je funkce rostoucí 
a na intervalu \(\langle-\infty,-\frac{1}{2}\rangle\) je klesající.

\begin{displaymath}
k(-\frac{1}{2})=-\frac{1}{2}^4 -2-\frac{1}{2}^3 + 2-\frac{1}{2} + 1
\end{displaymath}

\begin{displaymath}
k(-\frac{1}{2})=\frac{5}{16}
\end{displaymath}

Po dosazení našeho minima do funkce \(k\) zjistíme, že pod odmocninou bude minimální hodnota \(\frac{5}{16}\), tudíž jmenovatel nikdy nebude nulové nebo nedefinované hodnoty.

Čitatele si upravíme pomocí Hornerova schématu, aby se nám snadněji zjišťovaly snadněji nulové body.

Vzhledem k předchozímu zjištění, že jmenovatel bude vždy nabývat nenulové a definované hodnoty, můžeme jmenovatel zanedbat při hledání minima funkce \(f\).\\

Pro úpravu výrazu opět použijeme Hornerovo schéma.

\begin{table}[!h]
\centering
\begin{tabular}{l||l|l|l|l|l}
	 & 2 & -3 &  0 & 1 &    \\ \hline\hline
   1 & 2 & -1 & -1 & 0 & OK \\ \hline
   1 & 2 &  1 &  0 &   & OK \\
\end{tabular}
\caption{Hornerovo schéma pro rozklad $2x^3 - 3x^2 + 1$}
\end{table}

\begin{displaymath}
f(x)=(x-1)^2\cdot (2x +1)
\end{displaymath}

Podezřelé body: 1, \(-\frac{1}{2}\)

\begin{figure}[H]
	\centering
	\includesvg[width=.45\textwidth]{assets/osa}
	\caption{Monotónost funkce \(f\)}
\end{figure}

Opět když ověříme monotónnost funkce, tak zjistíme, že v $x=1$ je pouze inflexní bod a v $x=-\frac{1}{2}$ minimum, 
které můžeme označit za globální, protože na intervalu \(\langle-\frac{1}{2},+\infty\rangle\) je funkce rostoucí 
a na intervalu \(\langle-\infty,-\frac{1}{2}\rangle\) je klesající.\\
Nyní dosadíme do naší původní funkce, abychom našli souřadnici na ose \(y\) hledaného bodu.

\begin{displaymath}
f(-\frac{1}{2})= x^2 - x
\end{displaymath}

\begin{displaymath}
f(-\frac{1}{2})= \frac{3}{4}
\end{displaymath}

{\Large Nejbližší bod na funkci $f$ k bodu A je bod $\big[-\frac{1}{2}, \frac{3}{4}\big]$.}
