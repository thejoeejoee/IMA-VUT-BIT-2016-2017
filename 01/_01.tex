\section{1. úkol}

\subsection{Zadání}
Rozložte na parciální zlomky tuto racionální lomenou funkci:

\begin{displaymath}
f(x)=\frac{3x^3+x^2-4x+16}{x^5+5x^4+9x^3+13x^2+14x+6}
\end{displaymath}

\subsection{Rozbor příkladu}
Máme za úkol najít rozklad na parciální zlomky. Polynom ve \textbf{jmenovateli má vyšší stupeň}, než polynom v čitateli. Není třeba provádět dělení a můžeme rovnou přistoupit k rozkladu.

Pro rozklad polynomu ve jmenovateli použijeme Hornerovo schéma a následně si napíšeme rovnici vyjadřující rozklad na jednotlivé parciální zlomky v obecném tvaru. Dle rovnice si poté sestavíme soustavu rovnic pro výpočet jednotlivých koeficientů. Řešením rovnice budou koeficienty z množiny $\mathbb{Q}$, čímž získáme rozklad funkce na parciální zlomky.

\subsection{Řešení}

Rozklad čitatele za pomocí Hornerova algoritmu na součin závorek.

\begin{table}[htb]
\centering
\begin{tabular}{l||l|l|l|l|l|l|l}
	 & 1 & 5 & 9 & 13 & 14 & 6 &    \\ \hline\hline
  -1 & 1 & 4 & 5 & 8  & 6  & 0 & OK \\ \hline
  -1 & 1 & 3 & 2 & 6  & 0  &   & OK \\ \hline
  -3 & 1 & 0 & 2 & 0  &    &   & OK \\
\end{tabular}
\caption{Rozklad čitatele}
\end{table}

Rozklad jmenovatele na součin v oboru reálných čísel je

\begin{displaymath}
	x^5+5x^4+9x^3+13x^2+14x+6 = (x+1)^2(x+3)(x^2+2)
\end{displaymath}

Výraz $x^2+2$ nelze dále v oboru reálných čísel rozložit. Dostáváme funkci:

\begin{displaymath}
f(x)=\frac{3x^3+x^2-4x+16}{(x+1)^2(x+3)(x^2+2)}
\end{displaymath}

Funkci můžeme nyní rozložit na parciální zlomky. Rozklad vypadá tedy následovně:

\begin{displaymath}
f(x)=\frac{3x^3+x^2-4x+16}{(x+1)^2(x+3)(x^2+2)}
=
\frac{A}{x+1} +
\frac{B}{(x+1)^2} +
\frac{C}{x+3} +
\frac{Dx+E}{x^2+2}
\end{displaymath}

Rovnici upravíme:
\begin{gather*}
\begin{aligned}
3x^3+x^2-4x+16 &= 
 A(x^4+4x^3+5x^2+8x+6) + \\
& + B(x^3+3x^2+2x+6) + \\
& + C(x^4+2x^3+3x^2+4x+2) +\\
& + (Dx+E)\cdot(x^3+5x^2+7x+3)
\end{aligned}
\end{gather*}

a po roznásobení dostáváme:

\begin{gather*}
 \begin{aligned}
3x^3 + x^2 - 4x + 16 &
  = A(x^4 + 4x^3+5x^2+8x+6) + \\
& + B (x^3+3x^2+2x+6) + \\
& + C (x^4+2x^3+3x^2+4x+2) + \\
& + D (x^4+5x^3+7x^2+3x) + \\
& + E (x^3+5x^2+7x+3)
	\end{aligned}
\end{gather*}

Vytkneme mocniny:

\begin{gather*}
\begin{aligned}
3x^3+x^2-4x+16 &
 = x^4(A+C+D) + \\
& \,+ x^3(4A+B+2C+5D+E) + \\
& \,+ x^2(5A+3B+3C+7D+5E) + \\
& \,+ x^1(8A+2B+4C+3D+7E) + \\
& \,+ x^0(6A+6B+2C+3E) 
\end{aligned}
\end{gather*}

a dle této rovnice sestavíme soustavu rovnic pro výpočet koeficientů $A, B, C, D$ a $E$:

\begin{gather*}
\begin{aligned}
0 &= A + C + D\\
3 &= 4A + B + 2C + 5D + E\\
1 &= 5A + 3B + 3C + 7D + 5E\\
-4 &= 8A + 2B + 4C + 3D + 7E\\
16 &= 6A + 6B + 2C + 3E
\end{aligned}
\end{gather*}

Vyřešením soustavy rovnic dostáváme $A=1, B=3, C=-1, D=0, E=-2$. \\Výsledkem rozkladu je:

\begin{displaymath}
f(x)=
\frac{1}{x+1} +
\frac{3}{(x+1)^2} -
\frac{1}{x+3} -
\frac{2}{x^2+2}
\end{displaymath}
