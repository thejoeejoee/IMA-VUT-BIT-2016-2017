\section{1. úkol}

\subsection{Zadání}
Rozložte na parciální zlomky tuto racionální lomenou funkci.

\begin{displaymath}
f(x)=\frac{3x^3+x^2-4x+16}{x^5+5x^4+9x^3+13x^2+14x+6}
\end{displaymath}

\subsection{Rozbor příkladu}
Máme za úkol najít rozklad na parciální zlomky. Polynom ve jmenovateli ma vyšší stupneň, než polynom v čitateli. Není třeba provádět dělení, a můžeme rovnou přistoupit k rozkladu.

Pro rozklad polynomu ve jmenovateli použijeme hornerovo schéma. Poté si napíšeme rovnici vyjadřující rozklad na jednotlivé parciální zlomky v obecném tvaru. Podle rovnice si sestavíme soustavu rovnic pro výpočet jednotlivých koeficientů. Řešením rovnice budou koeficienty z množiny $Q$. Tím získáme rozklad funkce na parciální zlomky.

\subsection{Řešení}

Rozklad čitatele za pomoci hornerova schémata na součin závorek.

	\begin{table}[htb]
	\centering
	
	\begin{tabular}{|l|l|l|l|l|l|l|l|}
	\hline
		 & 1 & 5 & 9 & 13 & 14 & 6 &    \\ \hline
	  -1 & 1 & 4 & 5 & 8  & 6  & 0 & OK \\ \hline
	  -1 & 1 & 3 & 2 & 6  & 0  &   & OK \\ \hline
	  -3 & 1 & 0 & 2 & 0  &    &   & OK \\ 
	\hline
	\end{tabular}
	\caption{Rozklad čitatele}
	\end{table}


Rozklad jmenovatele na součin v oboru reálných čísel je tedy. 

\begin{displaymath}
	x^5+5x^4+9x^3+13x^2+14x+6 = (x+1)^2(x+3)(x^2+2)
\end{displaymath}

Výraz $x^2+2$ nelze dále v oboru reálných čísel rozložit. Dostáváme tedy funkci

\begin{displaymath}
f(x)=\frac{3x^3+x^2-4x+16}{(x+1)^2(x+3)(x^2+2)}
\end{displaymath}

Funkci můžeme nyní rozložit na parciální zlomky. Rozklad bude vypadat následovně.

\begin{displaymath}
f(x)=\frac{3x^3+x^2-4x+16}{(x+1)^2(x+3)(x^2+2)}
=
\frac{A}{x+1} +
\frac{B}{(x+1)^2} +
\frac{C}{x+3} +
\frac{Dx+E}{x^2+2}
\end{displaymath}

Rovnici upravíme na tvar

\begin{displaymath}
3x^3+x^2-4x+16 = A(x^4 + 4x^3+5x^2+8x+6) + B(x^3+3x^2+2x+6) + C(x^4+2x^3+3x^2+4x+2) + 
\end{displaymath}
\begin{displaymath}
+(Dx+E)\cdot(x^3+5x^2+7x+3)
\end{displaymath}

Po roznásobení závorky dostáváme

\begin{displaymath}
3x^3+x^2-4x+16 = A(x^4 + 4x^3+5x^2+8x+6) + B(x^3+3x^2+2x+6) + C(x^4+2x^3+3x^2+4x+2) +
\end{displaymath}
\begin{displaymath}
+D(x^4+5x^3+7x^2+3x) + E(x^3+5x^2+7x+3)
\end{displaymath}

Rovnici si upravíme vytknutím mocnin

\begin{displaymath}
3x^3+x^2-4x+16 = x^4(A+C+D) + x^3(4A+B+2C+5D+E) + x^2(5A+3B+3C+7D+5E) + 
\end{displaymath}
\begin{displaymath}
+x(8A+2B+4C+3D+7E) + (6A+6B+2C+3E)
\end{displaymath}

Podle této rovnice si sestavíme soustavu rovnic pro výpočet koeficientů $A, B, C, D$ a $E$

\begin{displaymath}
0 = A + C + D
\end{displaymath}
\begin{displaymath}
3 = 4A + B + 2C + 5D + E
\end{displaymath}
\begin{displaymath}
1 = 5A + 3B + 3C + 7D + 5E
\end{displaymath}
\begin{displaymath}
-4 = 8A + 2B + 4C + 3D + 7E
\end{displaymath}
\begin{displaymath}
16 = 6A + 6B + 2C + 3E
\end{displaymath}

Vyřešením soustavy rovnic dostáváme $A=1, B=3, C=-1, D=0, E=-2$ Funkce rozložená na parciální zlomky má tedy tvar.

\begin{displaymath}
f(x)=
\frac{1}{x+1} +
\frac{3}{(x+1)^2} -
\frac{1}{x+3} -
\frac{2}{x^2+2}
\end{displaymath}
