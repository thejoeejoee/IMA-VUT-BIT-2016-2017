\section{1. úkol}

\subsection{Zadání}
Rozložte na parciální zlomky tuto racionální lomenou funkci.

\begin{displaymath}
f(x)=\frac{3x^3+x^2-4x+16}{x^5+5x^4+9x^3+13x^2+14x+6}
\end{displaymath}

\subsection{Rozbor příkladu}
Máme za úkol najít rozklad na parciální zlomky. Polynom ve jmenovateli ma vyšší stupneň, než polynom v čitateli. Není třeba provádět dělení, a můžeme rovnou přistoupit k rozkladu.

Pro rozklad polynomu ve jmenovateli použijeme hornerovo schéma. Poté si napíšeme rovnici v obecném tvaru, kde v čitateli 

\subsection{Řešení}

\noindent Rozklad čitatele za pomoci hornerova schémata

\addvbuffer[12pt 8pt] {
	\begin{tabular}{|l|l|l|l|l|l|l|l|}
	\hline
		 & 1 & 5 & 9 & 13 & 14 & 6 &    \\ \hline
	  -1 & 1 & 4 & 5 & 8  & 6  & 0 & OK \\ \hline
	  -1 & 1 & 3 & 2 & 6  & 0  &   & OK \\ \hline
	  -3 & 1 & 2 & 0 &    &    &   & OK \\ 
	\hline
	\end{tabular}
}

Rozklad jmenovatele na součin v oboru reálných čísel je tedy.

\begin{displaymath}
	x^5+5x^4+9x^3+13x^2+14x+6 = (x+1)^2(x+2)(x^2+2)
\end{displaymath}

Výraz $x^2+2$ nelze dále v oboru reálných čísel rozložit. Dostáváme tedy funkci

\begin{displaymath}
f(x)=\frac{3x^3+x^2-4x+16}{(x+1)^2(x+2)(x^2+2)}
\end{displaymath}

Funkci můžeme nyní rozložit na parciální zlomky. Rozklad bude vypadat následovně.

\begin{displaymath}
f(x)=\frac{3x^3+x^2-4x+16}{(x+1)^2(x+2)(x^2+2)}
=
\frac{A}{x+1} +
\frac{B}{(x+1)^2} +
\frac{C}{x+2} +
\frac{Dx+E}{x^2+2}
\end{displaymath}


\todo{1. úkol}